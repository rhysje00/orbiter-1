\subsection{Finite Fields}


Suppose we want to have a finite field of order $q$. The \verb'finite_field' class 
implements a lot of algorithms for finite fields. Let us postpone the discussion of how the field is implemented and look at an example first.
Suppose we want the field with $4$ elements.

{\tt
\input CODE/finite_field.tex
}

The line \verb'finite_field *F;' defines a pointer to an object \verb'F' 
of type \verb'finite_field'. The line \verb'F = new finite_field;' allocates the object.
The line \verb'F->init(q, verbose_level);' initializes the finite field of order \verb'q' (here, 4).
The second parameter \verb'verbose_level' is indicating how much the initialization function should print. 
Higher numbers give more print. Passing zero means do not print at all.
The program produces the following output:

{\tt
\input CODE/finite_field_out.tex
}

If we change \verb'verbose_level' to 1 
and run the program again, we find the following output:

{\tt
\input CODE/finite_field_out1.tex
}

This reveals the polynomial that is used to create the finite field. 
In this example, we use the polynomial 
$x^2+x+1.$ It is very easy to use a different polynomial. 
If we want to create the finite field 
$\bbF_q$ over $\bbF_p$, we require an irreducible 
polynomial $m(x) = \sum_{i=0}^{n}a_i x^i \in \bbF_p[x]$ with $a_i \in \bbF_p.$
Using the convention that $0 \le a_i < p$ for all $i$, we can define the integer 
$m(p) = \sum_{i=0}^{n}a_i p^i.$ 
The function 
\begin{quote}
\verb'finite_field::init_override_polynomial(INT q,' \\
\quad\verb'const BYTE *poly, INT verbose_level);'
\end{quote}
can be used to define a finite field using the polynomial $m(p).$ 
Here, \verb'poly' points to a string which represents $m$ in decimal notation. 
So, for instance,  \verb'init_override_polynomial(4, "7", verbose_level)' creates the same field as before,  
using the polynomial $x^2+x+1,$ since $7 = 2^2+2+1.$

\smallskip


How are the elements of the finite field encoded? If $q=p$ is prime, the elements of $\bbF_p$ are the integers 
$\{0,1,\ldots,p-1\}.$ If $q = p^h$ for some prime $p$, and some integer $h > 1,$ 
the assumption is made that the polynomial $m(x)\in \bbF_p[x]$ 
used to create $\bbF_q$ is primitive. That is, a root $\alpha$ of $m(x)$ is a generator for the multiplicative 
group $\bbF_q^\times.$ 
Every element in $\bbF_q$ is of the form $w(\alpha)$ where 
$w(x) = \sum_{i=0}^{h-1} w_i x^i \in \bbF_p[x]$ is a polynomial of degree 
at most $h-1.$ The element $w(\alpha)$ is represented by the integer $w(p) = \sum_{i=0}^{n}w_i p^i.$ 
Thus $0$ is represented by $0$ and $1$ is represented by 1. 
The primitive element $\alpha$ is represented by $p$.





